\documentclass[a4,12pt]{amsart}
%%%%%%%%%%%%%%%%%%%%%%%%%%%%%%%%%%%%%%%%%%%%%%%%%%%%%%%%
\oddsidemargin 0mm
\evensidemargin 0mm
\topmargin 0mm
\textwidth 160mm
\textheight 230mm
\tolerance=9999
%%%%%%%%%%%%%%%%%%%%%%%%%%%%%%%%%%%%%%%%%%%%%%%%%%%%%%%%
\usepackage{amssymb,amstext,amsmath,amscd,amsthm,amsfonts,enumerate,graphicx,latexsym, hyperref,cleveref,bm} 
\usepackage[T1]{fontenc}
%\usepackage[czech]{babel}
%\babelprovide[import,main]{czech}
\usepackage[usenames]{color}
%\usepackage{showkeys}
\usepackage[all]{xy}
\newcommand{\cmt}[1]{{\color{red} #1}}
\newcommand{\new}[1]{{\color{blue} #1}}
 \usepackage{comment}
 \usepackage{mathrsfs}
 \usepackage{tikz-cd}
\usepackage{relsize}
\usepackage{mathtools}


\numberwithin{equation}{section}
\newtheorem{lemma}[equation]{Lemma}
\newtheorem{lem}[equation]{Lemma}
\newtheorem{theorem}[equation]{Theorem}
\newtheorem*{gr}{Green's Linear Syzygy Theorem}
\newtheorem{propo}[equation]{Proposition}
\newtheorem{prop}[equation]{Proposition}
\newtheorem{cor}[equation]{Corollary}
\newtheorem{conj}[equation]{Conjecture}
\newtheorem{claim}[equation]{Claim}
\newtheorem{claim*}{Claim}
\newtheorem{thm}[equation]{Theorem}
\newtheorem{convention}[equation]{Convention}
\newtheorem{question}[equation]{Question}


\theoremstyle{definition}
\newtheorem{defn}[equation]{Definition}
\newtheorem{dfn}[equation]{Definition}
\newtheorem{notation}[equation]{Notation}
\newtheorem{example}[equation]{Example}
\newtheorem{ex}[equation]{Example}
\newtheorem{construction}[equation]{Construction}
\newtheorem{constr}[equation]{Construction}
\newtheorem{algorithm}[equation]{Algorithm}
\newtheorem{warning}[equation]{Warning}
\newtheorem{conventions}[equation]{Conventions}
\newtheorem{setup}[equation]{Setup}
\newtheorem{nota}[equation]{Notation}
\newtheorem{conv}[equation]{Convention}


\theoremstyle{remark}
\newtheorem{remark}[equation]{Remark}
\newtheorem{remarks}[equation]{Remarks}
\newtheorem{rem}[equation]{Remark}

\newtheorem*{ac}{Acknowlegments}

\newcommand{\projdim}{\operatorname{proj\,dim}}
%%%%%%%%%%%%%%%%%%%%%%%%%%%%%%%%%%%%%%%%%%%%%%%%%%%%%%%%%%%%%%%%%
\renewcommand{\qedsymbol}{$\blacksquare$}
\numberwithin{equation}{section}
%%%%%%%%%%%%%%%%%%%%%%%%%%%%%%%%%%%%%%%%%%%%%%%%%%%%%%%%%%%%%%%%%
\def\ann{\operatorname{ann}}
\def\ass{\operatorname{Ass}}
\def\assh{\operatorname{Assh}}
\def\cm{\operatorname{CM}}
\def\codim{\operatorname{codim}}
\def\cok{\operatorname{Coker}}
\def\depth{\operatorname{depth}}
\def\E{\operatorname{E}}
\def\End{\operatorname{End}}
\def\Ext{\operatorname{Ext}}
\def\height{\operatorname{ht}}
\def\h{\operatorname{H}}
\def\Hom{\operatorname{Hom}}
\def\I{\operatorname{I}}
\def\id{\mathrm{id}}
\def\im{\operatorname{im}}
\def\jac{\operatorname{Jac}}
\def\ker{\operatorname{ker}}
\def\lend{\operatorname{\underline{End}}}
\def\lhom{\operatorname{\underline{Hom}}}
\def\lmod{\operatorname{\underline{mod}}}
\def\m{\mathfrak{m}}
\def\Min{\operatorname{Min}}
\def\mod{\operatorname{mod}}
\def\Mod{\operatorname{Mod}}
\def\n{\mathfrak{n}}
\def\ng{\operatorname{NG}}
\def\p{\mathfrak{p}}
\def\res{\operatorname{res}}
\def\soc{\operatorname{soc}}
\def\spec{\operatorname{Spec}}
\def\sing{\operatorname{Sing}}
\def\syz{\Omega}
\def\t{\mathrm{t}}
\def\Tor{\operatorname{Tor}}
\def\Tr{\operatorname{Tr}}
\def\tr{\operatorname{tr}}
\def\uend{\operatorname{\overline{End}}}
\def\uhom{\operatorname{\overline{Hom}}}
\def\V{\operatorname{V}}
\def\type{\operatorname{type}}
\def\cmf{\operatorname{CM_{full}}}
\def\supp{\operatorname{Supp}}
\def\chr{\operatorname{char}}
\def\G{\operatorname{\textbf{G}}}
\def\sh{\operatorname{Sh}}
\def\U{\operatorname{\textbf{U}}}
\def\T{\operatorname{\textbf{T}}}
\def\A{\operatorname{\textbf{A}}}
\def\K{\operatorname{\textbf{K}}}
\def\bar{\operatorname{\textbf{Bar}}}
\def\syz{\operatorname{Syz}}
\def\Syz{\operatorname{Syz}}
\def\det{\operatorname{\textbf{Det}}}
\def\sing{\operatorname{\mathrm{Sing}}}
\def\rk{\operatorname{rank}}
\def\a{\operatorname{\textbf{a}}}
\def\u{\operatorname{\textbf{u}}}
\def\ca{\operatorname{\mathrm{ca}}}
\def\cone{\operatorname{\mathrm{\textbf{Cone}}}}
\def\I{\operatorname{\mathbf{I}}}
\def\itot{\operatorname{\mathbf{I}^{tot}}}
\def\istab{\operatorname{\mathbf{I}^{stab}}}
\def\d{\operatorname{\boldsymbol{\delta}}}
\def\M{\operatorname{\boldsymbol{M}}}
\def\C{\operatorname{\mathrm{C}}}
\def\grade{\operatorname{\mathrm{Grade}}}
\def\F{F}
\def\ord{\operatorname{ord}}
\def\hilb{\operatorname{\boldsymbol{H}}}
\def\cond{\mathfrak{c}}
\def\cx{\operatorname{\boldsymbol{cx}}}
\newcommand{\ses}[3]{0 \to {#1} \to {#2} \to {#3} \to 0}
\newcommand{\ds}{\displaystyle}
%macros added by MB
\def\nc{\newcommand}
\def\on{\operatorname}
\def\th{\on{th}}
\nc{\from}{\leftarrow}
\nc{\xra}{\xrightarrow}
\nc{\xla}{\xleftarrow}
\def\:{\colon}
\nc{\tQ}{\widetilde{Q}}
\nc{\tf}{\widetilde{f}}
\nc{\Db}{\on{D}^{\on{b}}}
\nc{\tD}{\widetilde{D}}
\nc{\coker}{\on{coker}}
\def\Z{\mathbb{Z}}
\nc{\odd}{\on{odd}}
\nc{\even}{\on{even}}
\nc{\tS}{\widetilde{S}}
\nc{\mf}{\on{mf}}
\nc{\hmf}{\on{hmf}}
\nc{\op}{\on{op}}
\nc{\pd}{\on{pd}}
\nc{\Perf}{\on{Perf}}
\nc{\st}{\on{st}}
\def\MR#1{}
\nc{\into}{\hookrightarrow}
\nc{\onto}{\twoheadrightarrow}
\def\top{\on{top}}
\nc{\Jac}{\on{Jac}}

%%%%%%%%%%%%%%%%%%%%%%%%%%%%%%%%%%%%%%%%%%%%%%%%%%%%%%%%%%%%%%%%%
\begin{document}
%\allowdisplaybreaks
\setlength{\baselineskip}{14pt}
\title{Notes on infinite free resolutions}
\author{Michael K. Brown}


\author{Hailong Dao}

\author{David Eisenbud}
\author{Prashanth Sridhar}


\newcommand{\Addresses}{{
	\vskip\baselineskip
  	\footnotesize
  	\noindent \textsc{Department of Mathematics and Statistics, Auburn University} \par\nopagebreak
	\noindent \textit{E-mail addresses:} \texttt{mkb0096@auburn.edu, prashanth.sridhar0218@gmail.com}
  	\vskip\baselineskip
   \noindent \textsc{Department of Mathematics, University of California, Berkeley} \par\nopagebreak
	\noindent \textit{E-mail address:} \texttt{de@berkeley.edu}
  	\vskip\baselineskip
  	\noindent \textsc{Department of Mathematics, University of Kansas} \par\nopagebreak
	\noindent \textit{E-mail address:} \texttt{hdao@ku.edu}
}}


%\begin{abstract}
%\end{abstract}
\keywords{} 
\subjclass[2010]{}
\maketitle
%\tableofcontents
%%%%%%%%%%%%%%%%%%%%%%%%%%%%%%%%%%%%%%%%%%%%%%%%%%


%%%%%%%%%%%%%%%%%%%%%%%%%%%%%%%
\section{Ideas}
%%%%%%%%%%%%%%%%%%%%%%%%%%%%%%%

\begin{enumerate}


\item We know periodicity of ideals of minors does not hold over Koszul algebras (Gasharov-Peeva give a counterexample). But can we compute the stable ideals of $1 \times 1$-minors? It seems that, generically, we get the maximal ideal. 
\item A question related to (2): can we show that the linear part of any minimal free resolution of a module over a Koszul algebra is eventually exact, as is the case over an exterior algebra (see \cite[Theorem 3.1]{EFS})?

\item Can we prove/disprove periodicity of ideals of minors for 1-dimensional Gorenstein algebras? What (more) can we say about the stable ideals of minors in this case?

\item Eisenbud-Reeves-Totaro have conjectured that, given a finitely generated module $M$ over a standard graded polynomial ring, there exists a choice of basis of the minimal free resolution $F$ of $M$ and some $N \ge 0$ such that the entries of the matrices in $F$ have degree at most $N$. It seems we can prove this over complete intersections and Golod rings. Is this worth writing up? What is the next natural family of examples to consider?

\item Can we compute stable ideals of minors over hypersurface rings?

\item Socle summands: In all examples seen so far (June 30,2023) the semigroup of positions in which socle summands occur
in the syzygies of the residue field of an artin local ring
consists of all integers starting with a some number $d$, ie $d+NN$ OR $(n+1, n+3+NN$ where n is the embedding dimension. This is true for Burch rings ($d=2$ and for
Golod ring (where both patters occur).

\item Moving beyond socle summands: when do syzygy modules of indecomposable modules (eg of the residue field) decompose as direct sums, either over the ground field or over the algebraic closure? eg: never for 0-dimensional Gorenstein rings. This can be investigated via the endomorphism rings of syzygies. (Sayrafi has begun some coding to detect idempotents).
\end{enumerate}

We also have:

\begin{question}[Dao-Eisenbud]\label{DE_conjecture}
    Given a finitely generated module $M$ over a local ring $R$, must there be a number $n(M)$ such that the sum of the ideals of $1\times 1$ minors in $n(M)$ consecutive differential matrices of the minimal resolution of $M$ eventually becomes constant? Is there an integer only depending on $R$ that bounds all $n(M)$ from above?


\end{question}

In fact we know no example of a ring and a module $M$ where we cannot take $n(M) = 2$. [Doesn't the material in \cite{DMS} prove that
n=2 works for complete intersections and Golod rings?]

 \section{Stability of ideal of minors up to radical}

 Throughout this section $R$ will be a (graded) local ring, $M$ a finite module over it and $\F\xra{\simeq} M$ it's minimal (graded) free resolution.

\par 

 \subsection*{Proposed strategy for approaching \Cref{DE_conjecture} up to radical in some cases:} 
 The up to radical version of the Dao -Eisenbud conjecture states that if a prime ideal contains the sum of ideal of entries of $n(M)$ consecutive matrices, then it contains that of the differentials prior and after. Suppose these $n(M)$ matrices are the beginning of the minimal free resolution of $N$. Then if $P$ contains the sum of ideal of entries of $n(M)$ consecutive matrices, then the first $n(M)-1$ matrices localized at P can be completed to a minimal free resolution of $N_P$. If the ring $R$ and $R_P$ are such that for any module, one can determine all Betti numbers, given finitely many of them (ex: rational Poincare series), then one can try to bound $n(M)$ using this.
 
If $R$ is Gorenstein one can dualize the complex if $N$ is MCM and use this approach to address the matrix that appears prior.

In the opposite direction, there are some rings where the Betti numbers
of every module are strictly increasing. If this is true for
a localization $R_P$, then $P$ must contain minors of a certain size.

\subsection*{Other miscellaneous notes:}

\begin{rem}
    I find the convention that $I_1(0) = 1$ a little strange, though it does agree with the idea the the top fitting ideal is 1 iff the cokernel is projective\dots 
\end{rem}
\begin{rem}\label{rem: reg_sequence_ideal_of entries}
    Suppose $R$ is Cohen-Macaulay of dimension $d$. Then $\I^1_i(M)\not\subseteq (\underline{x})$ for any regular sequence $\underline{x}$ in $R$ and for all $i\geq d+1$. This is clear if $M$ has finite projective dimension, so assume $M$ has infinite projective dimension. Since $\syz_d(M)$ is maximal Cohen-Macaulay, any regular sequence $\underline{x}$ of $R$ is also regular on $\syz_d(M)$. Thus, $\F_{\geq d}\otimes_R R/(\underline{x}R)$ is a minimal acyclic complex and hence none of it's differentials can be the zero map.

    Here the ``regular sequence'' could be replaced with the annihilator of any module of finite projective dimension, trading concreteness for generality (I usually prefer concreteness!)
\end{rem}

\begin{question}
    Are there constraints similar to those in \Cref{rem: reg_sequence_ideal_of entries} for higher rank minors?
\end{question}

 \begin{lem}\label{lem: no_free_summands}
     Let $M$ be non-free such that $\depth(M)\geq \depth(R)$. Then $\syz_1(M)$ has no free summands.
 \end{lem}
 \begin{proof}
     Note that $\projdim(M)=\infty$. Consider the exact sequence $\G$
     
$$
  0 \xla{}  M \xla{} F_0 \xla{} \syz_1(M) \xla{} 0
 $$


 Note that $\syz_1(M)$ has depth at least $\depth(R)$. By prime avoidance, we can choose a maximal regular sequence $\underline{x}$ in $R$ that is also regular on the flanking modules. Thus, $\G\otimes_R R/(\underline{x})R$ is exact. Moreover, $\F\otimes_R R/(\underline{x})R$ is acyclic. Note that $\projdim_{R/(\underline{x})R}M/(\underline{x})M = \projdim_R(M)=\infty$. We have $\syz_1(M)\otimes_R R/(\underline{x})R\simeq \syz_1^{R/(\underline{x})R}(M\otimes_R R/(\underline{x})R)$. Since the latter is annihilated by the socle (which is non-zero) of $R/(\underline{x})R$, it cannot have a free summand over $R/(\underline{x})R$ and hence $\syz_1(M)$ cannot have a free summand over $R$.
 
 %is free over $R/(\underline{x})$ if and only if it is free over $R/(\underline{x})$ of rank equal to $\rk_R(\F_0)$ if and only if $\syz_1(M)/(\underline{x})\syz_1(M)=0$. The last assertion is impossible by Nakayama and hence $\projdim_{R/(\underline{x})R}M/(\underline{x})M=\infty$. 

 \end{proof}

 \begin{cor}
     Assume $\projdim_R(M)=\infty$. For all $n\geq \depth(R)+1$, $\syz_n(M)$ does not have a free summand.
 \end{cor}
 \begin{proof}
     This is immediate from \Cref{lem: no_free_summands}.
 \end{proof}

 \begin{rem}\label{rem: BH}
     Suppose we have a map of finite free $R$-modules $\phi:F\rightarrow G$. Then, for $P\in Spec(R)$, $\I^1(\phi)\not\subseteq P$ implies $\im(\phi_P)$ contains a free summand of $G_P$ (see for example \cite[Lemma 1.4.8]{bruns_herzog}). After a change of basis of $G_P$, one sees that this in turn implies that $\im(\phi_P)$ has a free summand.
 \end{rem}



 %\subsection{Note on $1$-periodicity}

%Here we note that in general we cannot expect anything less than $2$-periodicity of ideal of minors in the setting of \Cref{CI_periodicity}.  Starting with codimension one, this is clearly too much to expect : the matrix factorization $(x,x^2)$ of $x^3$ in $k[x]$ gives a counterexample. What if we restrict attention to \textit{irreducible} hypersurfaces ? Let us fix some notation. Let $S$ be a regular local ring, $f\in S$ irreducible and set $R\coloneqq S/(f)$. Let $(d,h)$ be a minimal matrix factorization for $f$ of equal rank:

%\begin{tikzpicture}  
 %[every node/.style={scale=1},  auto]
%\node(20){$S^a$};
%\node(21)[node distance=2.5cm, right of=20]{$S^a$};
%\node(22)[node distance=2.5cm, right of=21]{$S^a$};
%\node(23)[node distance=2.5cm, right of=22]{$S^a$};
%\node(30)[node distance=2.5cm, below of=20]{$A_{1}(p)/A_{1}(p-1)$};
%\node(32)[node distance=2.5cm, below of=22]{$A_{1}(p)/A_{1}(p-1)$};
%\draw[->] (20) to node[swap] {$\pi_p$} (30);
%\draw[->] (22) to node {$\pi_p$} (32);
%\draw[->] (20) to node [above=1pt] {$d$} (21);
%\draw[->] (21) to node [above=1pt]{$h$} (22);
%\draw[->] (22) to node [above=1pt] {$d$} (23);
%\draw[->, bend right=40, ] (21)  to node[above=0pt]  {$f$} (23);
%\draw[->, bend left=40, ] (20)  to node[above=0pt]  {$f$} (22);
%\draw[->, bend left=40, ] (30)  to node[above=0pt] {$f_{p}$} (32);
%\end{tikzpicture}
%\end{center}
%Since $f$ is irreducible, $\det(d)=f^m$ and $\det(h)=f^n$ for some non negative integers $m,n$ such that $m+n=a$.
%Let $d_f$ and $h_f$ denote the natural images of $d$ and $h$ respectively upon inverting $f$. We have $d_f(\dfrac{1}{f}h_f)=\id_{S_f^a}$. If $A$ is an invertible matrix with entries in a commutative ring, the ``Jacobi identity" or ``Schur complement formula" says that if $I$ and $J$ are collections of rows and columns of $A$ of the same cardinality, then
%\[\det(A[I,J])=\det(A)\det(A^{-1}[J^c,I^c])\]
%Here $A[I,J]$ denotes the submatrix of $A$ obtained by only retaining the rows from $I$ and the columns from $J$, while $I^c$ (resp. $J^c$) denotes the collection of rows (resp. columns) of $A$ that do not lie in $I$ (resp. $J$). From this relation, we see that $\I^r(d\otimes R)=0$ for $r>n$ and $\I^r(h\otimes R)=0$ for $r>m$. Moreover, $\I^n(d\otimes R)=\I^m(h\otimes R)$. Without loss of generality, assume that $m\leq n$. Suppose that $1$-periodicity of ideal of minors of all sizes holds for the minimal free resolution of $\cok(d\otimes R)$.  We have in particular that $\I^n(h\otimes R)=\I^n(d\otimes R)=\I^m(h\otimes R)$. If $m>n$, then by Nakayama's lemma, $\I^m(h\otimes R)=0$, so that $\rk(h\otimes R)<m$. This gives
%\[a=\rk(h\otimes R)+\rk(d\otimes R)<m+n=a\] A contradiction. Therefore $m=n$ is necessary for 1-periodicity - in other words, $\rk(d\otimes R)=\rk(h\otimes R)$ is necessary for $1$-periodicity. In particular, matrix factorizations of odd sizes of irreducible hypersurfaces cannot possess $1$-periodicity. One may wonder if the converse is true: that is, does every matrix factorization of equal rank over an irreducible hypersurface possess $1$-periodicity of ideal of minors of all sizes? This is not true, as the following example from \cite{yoshino_1990} illustrates.
%\par Take $S=k[x,y]$ for $k$ a field and $f=x^3+y^5$. Take

%$$
%d = \begin{pmatrix}
%x & y^2 & 0 & y  \\
%y^3 & -x^2 & -xy^2 & 0 \\
%0 & 0 & x^2 & y^2  \\
%0 & 0 & y^3 & -x 
%\end{pmatrix},
%\quad 
%h = \begin{pmatrix}
%x^2 & y^2 & 0 & xy \\
%y^3 & -x & -y^2 & 0 \\
%0 & 0 & x & y^2 \\
%0 & 0 & y^3 & -x^2
%\end{pmatrix},
%$$
%Then $\det(d)=\det(h)=f^2$, but $\I^1(d\otimes R)=(x,y)R$ and $\I^1(h\otimes R)=(x,y^2)R$.



\bibliography{references}
\bibliographystyle{amsalpha}
\Addresses

\end{document}




